\documentclass[a4paper, 11pt]{article}

\usepackage[utf8]{inputenc}
\usepackage[T1]{fontenc}
\usepackage[francais]{babel} 
\usepackage{amsmath} % pour les formules de maths
\usepackage{amssymb} % pour des symboles
\usepackage{mathrsfs} % pour avoir acces a des jolies lettres calligrafiées. :)
\usepackage{listings} % pour le code source
\usepackage{color} % pour les couleurs
\usepackage{graphicx} % pour les graphiques (images)
\usepackage{fancyhdr} % pour utiliser le pagestyle fancy
\usepackage[headheight=10pt]{geometry} % pour les marges
\usepackage{hyperref}

\geometry{hmargin=3cm}

\title{Projet de Master}
\author{Romain Mencattini}
\date{\today}

\pagestyle{fancy} % pour avoir des entetes et des pieds de page
\renewcommand\headrulewidth{0.6pt}
\fancyhead[L]{Romain Mencattini} % haut de page gauche
\fancyhead[R]{Université de Genève \today} % haut de page droite

\begin{document}
\maketitle
\newpage
\tableofcontents
\newpage

%%%%%%%%%%%%%%%%%%%%%%%%%%%%%%%%%%%%%%%%%%%%%%%%%%%%%%%%%%%%%%%%%%%%%%%%%%
% début de l'état de l'art %%%%%%%%%%%%%%%%%%%%%%%%%%%%%%%%%%%%%%%%%%%%%%%
%%%%%%%%%%%%%%%%%%%%%%%%%%%%%%%%%%%%%%%%%%%%%%%%%%%%%%%%%%%%%%%%%%%%%%%%%%
\section{État de l'art}
\subsection{Introduction}
\paragraph{}
Avant la démocratisation de l'informatique et de son utilisation, toutes les opérations financières étaient réalisées par des humains. Ce système pouvait avoir des inconvénients :
\begin{itemize}
\item L'émotionnel pouvait influer les transactions. En effet, ces dernières étant effectuées par des humains, il y avait une risque non négligeable que l'état de la personne agisse sur sa décision.
\item Un problème sous-jacent était de maintenir une discipline de \textit{trading}. Afin de minimiser les pertes et de maximiser les gains, il fallait se tenir à ce plan afin de ne pas ce laisser influencer par des paramètres extérieurs. Cela pouvait être très difficile.
\item Le \textit{backtesting}\footnote{Backtesting is the process of testing a trading strategy on relevant historical data to ensure its viability before the trader risks any actual capital. \ref{backtesting investopedia}} était impossible. Tester la qualification ainsi que la qualité de \textit{trading} d'une personne était compliquée. De même pour un \textit{trading plan}.
\end{itemize}

\paragraph{}
Ces éléments ont, en partie, favorisé l'émergence et l'utilisation d'algorithmes dans la finance. En 2014 aux USA, 84\% des transactions étaient accomplies par des algorithmes \ref{real investors}. Ce qui représente environ 100000 réalisations, ou \textit{ticks}, par secondes \ref{real investors}.
De part l'utilisation intensive de cet outil informatique, le monde de la finance a suivi l'évolution de ce domaine. Afin de perfectionner leurs algorithmes.
On retrouve donc des méthodes d'optimisations poussées ainsi que les récentes découvertes de \textit{data mining} et de \textit{machine learning}, abrégé \textit{ML}. Des propositions de plus en poussées dans les deux domaines voient le jour. L'algorithme qui sera au coeur de ce projet en fait partie. Il s'agit d'un réseau de neurones avec plusieurs couches prenant en compte des paramètres particuliers à la finance.

\paragraph{}Afin d'approcher aux mieux ces notions, nous allons discuter des éléments nécessaires à la compréhension.
Nous allons en premier lieu introduire le domaine financier ainsi que ces outils. Puis nous parlerons de plusieurs méthodes de \textit{ML}.
Voici celles vont être développées dans cet état de l'art :
\begin{itemize}
\item Les réseaux de neurones.
\item Les arbres de décision.
\item Les algorithmes \textit{SVM} \ref{wikipedia svm}.
\item \textit{Logistic Regression}.
\item \textit{Naive Bayes}.
\item Descente du Gradient \ref{wikipedia descente du gradient} ainsi que sa version dite stochastique \ref{descente du gradient stochastique} .
\end{itemize}

Finalement, nous lierons les deux domaines en montrant comment adapter les modèles mathématiques de \textit{ML} pour les utiliser comme techniques de \textit{trading}.


\subsection{Finance}

\subsubsection{\textit{FOREX}}

\paragraph{}Afin de comprendre le fonctionnement du \textit{FOREX}, il est important de mentionner certaines décisions historiques. Ces dernières ont façonné le marché des devises actuel.

\paragraph{}
Jusqu'à la première guerre mondiale, le système en vigueur se base sur l'or, que l'on nomme l'étalon-or\footnote{Source : \ref{étalon-or à étalon-dollar}.}. S'en suit une période d'instabilité notamment dûe aux pertes occasionnées par la guerre, un après-guerre compliqué, la crise boursière de 1929 et la seconde guerre mondiale.

C'est au sortir de cette dernière, que la nécessité de "\textit{mettre en place une organisation monétaire mondiale et de favoriser la reconstruction et le développement économique des pays touchés par la guerre}" \ref{wikipedia bretten woods}, est apparue. Le but était également "\textit{d’aplanir les conflits économiques, reconnaissant par là les problèmes engendrés par les disparités économiques}" \ref{étalon-or à étalon-dollar}.

Plusieurs propositions furent proposées, mais ce fût celle de Harry Dexter White qu'on mit en place. Cette dernière prévoyait entre autre:
\begin{itemize}
\item le choix du Dollar américain comme étalon, ce dernier étant rattaché à l'or\footnote{Suspension l'équivalence or pour le dollar américain en août 1971 puis abandon définitif en mars 1973 \ref{wikipedia bretten woods}.}.
\item Création de la Banque internationale pour la reconstruction et le développement (BIRD) qui deviendra la banque mondiale.
\item Le Fond monétaire international (FMI).
\item Création de l'Organisation mondial du commerce\footnote{Ne verra le jour qu'en 1995 faute d'accord \ref{wikipedia bretten woods}.}.
\end{itemize}

On remarque que ces institutions sont toujours en activité, cela démontre l'importance de ces accords pour comprendre le système financier. Cela est également vrai pour le marché des taux de changes, où le USD est toujours utilisé entre deux échanges.
Il n'est en effet pas possible de faire CHF/EUR\footnote{Franc Suisse - Euro}. Le passage par le USD entre les deux est obligatoire; nous aurons donc CHF/USD\footnote{Franc Suisse - Dollar Américain} puis USD/EUR\footnote{Dollar Américain - Euro}

\paragraph{}
Le marché \textit{FOREX} porte sur les devises. La valeur d'une devise ne peut être exprimée qu'en fonction d'une autre. Par exemple 1 franc suisse vaut 1.05 euro.\footnote{Taux fictif utilisé pour l'exemple.}
La transaction porte donc sur deux monnaies comme CHF/EUR. On va vendre des francs suisses pour acheter euros ou l'inverse.
Le nom du marché vient d'ailleurs de ces échanges. On échange une monnaie contre une autre, c'est un \textit{FOreign EXchange}, ou \textit{FOREX}.

Il y a deux variations de la monnaie possibles :
\begin{itemize}
\item La monnaie peut subir une dépréciation.
\item La monnaie peut subir une appréciation.
\end{itemize}
Lorsque le prix d'une monnaie augmente par rapport à une monnaie étrangère augmente, on parle d'appréciation. Ainsi dans le cas contraire, on parlera d'une dépréciation.

La mondialisation a facilité ce marché. En effet, toutes les devises sont accessibles depuis n'importe où. Il devient donc possible d'avoir des marchés avec des devises plus exotiques.

\paragraph{}
Les principaux acteurs financiers sont \ref{marche des changes}:
\begin{itemize}
\item Les banques commerciales. Elles peuvent pratiquer des interventions directes car elles gèrent des dépôts et veulent opérer des transactions sur ces derniers. Il est également possible de réaliser le rôle d'intermédiaire financier.
\item Les entreprises. Ces dernières vont pratique des transactions directes, si elles disposent d'un accès aux marchés sinon via des banques.
\item Les institutions financières non-bancaires. On peut citer les fonds de pensions, les sociétés d'assurances ou les \textit{hedge funds}. Ce sont surtout dans un but de spéculation, d'arbitrage ou de couverture de risque qu'elles interviennent.
\item Les banques centrale. Il peut y avoir des interventions directes, dans le but de modifier l'appréciation de la monnaie.
\item Les ménages. Surtout dans une optique de voyage, d'achat ou de spéculation.
\end{itemize}

\paragraph{}
Henry Bourguinat a énoncé "\textit{la règle des trois unités}" qui correspondent aux unités de temps, de lieu et d'opérations et d'acteurs.
Le \textit{FOREX} répond à ces trois unités \ref{site fr forex}:
\begin{itemize}
\item Ce marché fonctionne 24h/24 et les transactions s'effectuent presque en continue.
\item Il fonctionne à l'échelle mondiale tout en étant décentralisé. De part l'évolution des technologies, l'information circule aisément malgré son statut.
\item L'uniformité des procédés ainsi que des produits est présente. Les acteurs malgré nationalité sont de même nature.
\end{itemize}

\paragraph{}
Il existe principalement deux horizon temporels : le \textit{spot} et le {forward}.

Le premier est également appelé "Le marché au comptant". Lorsque deux acteurs se mettent d'accord sur une transaction, cette dernière se réalise immédiatement\footnote{Valable en théorie, dans la réalité cela peut prendre du temps \ref{marche des changes}}.

Le second peut être nommé "Le marché à terme". L'accord est passé à un temps $T$ mais la transaction effective ne se réalise que dans le futur. Ce futur, ou maturité, peut être de plusieurs dizaines de jours, voir des années, soit $T + X$.

Il y a opérations réalisable sur le marché à terme. :
\begin{itemize}
\item Les \textit{swaps}. Ils consistent vendre une monnaie au comptant puis à la racheter à terme\footnote{Soit à $T+X$}.
\item Les \textit{futures/forwards}. La différence entre ces deux tient surtout à leur standardisation et leur mise en place. Cependant le principe reste le même : on réalise une opération (d'achat ou de vente) qui ne s'effectuera qu'à maturité.
\item Les \textit{options}. Dans ce cas, on achète/vend un droit. Par exemple, on peut vendre le droit d'acheter une monnaie à un certain prix. Le détenteur du droit peut choisir de ne pas l'exercer au contraire du \textit{futures/forwards}.
\end{itemize}


\subsection{Cadre théorique des algorithmes de \textit{Machine Learning}}
\subsubsection{Introduction}
\paragraph{}
T. Mitchell a donné une définition formelle \ref{mitchell}:
\begin{center}
"\textit{A computer program is said to learn from experience $E$ with respect to some class of tasks $T$ and performance measure $P$ is its performance at tasks in $T$, as measured by $P$, improves with experience $E$}"
\end{center}

On a donc une tâche $T$ à accomplir, où $T$ peut être de trier des images ou de reconnaître des motifs. La mesure de la réussite de cette tâche $T$ est nommée $P$. C'est-à-dire la qualité du résultat du programme pour la tâche donnée, $T$. Si le programme améliore son résultat $P$ pour la tâche $T$ grâce à de l'expérience $E$. Il s'agit d'un programme de \textit{machine learning}. L'expérience peut être vu comme une phase d'entraînement ou comme le fait de retenir les réponses après avoir accompli la tâche.

\paragraph{}
Il existe deux catégories d'apprentissage :
\begin{itemize}
\item L'apprentissage supervisé.
\item L'apprentissage non-supervisé.
\end{itemize}

Dans le cas du premier, on fournit au programme, un ensemble d'entraînement\footnote{Ou d'expérience, $E$}, qui contient des réalisations ainsi que le résultat voulu. Le programme va donc pouvoir utiliser ce savoir afin d'améliorer sa performance $P$.

Pour l'apprentissage non-supervisé, on fournit des données, mais sans le résultat voulu. C'est uniquement après avoir décidé d'une valeur qu'on va signifier au programme si cette dernière est correcte. On ne lui donnera jamais la valeur attendue. Il va donc, uniquement en sachant si ces réponses sont vraies ou fausses, améliorer son $P$.

Par exemple, on désire reconnaître un certain type de voiture à partir d'images, mettons une citroën 2CV. Dans le cas de l'apprentissage supervisé, nous allons fournir au programme un ensemble d'entraînement qui contient de nombreuses photos de voitures, ainsi que la marque des dites voitures. L'algorithme va donc travailler avec ces données.

Par contre dans le cas de l'apprentissage non-supervisé, le programme ne pourra utiliser que les photos, et après avoir retourné le résultat, nous lui dirons si c'est juste ou faux.

\paragraph{}
Concernant, l'ensemble d'entraînement, il y a des points à prendre en compte afin de minimiser les risques de sur-apprentissage\footnote{Le sur-apprentissage consiste à apprendre par coeur la tâche, plutôt que d'apprendre les principes pour réaliser la tâche.}, et de maximiser la qualité de nos données.
Pour ce faire il faut:
\begin{itemize}
\item Représenter la population générale. Donc si le but est du traitement de la langue, il faut que la propension et la répartition des mots soient les mêmes que ceux de la langue.
\item Contenir des membres de chaque classes. Pour reconnaître des chiffres, il est important de disposer de chacun des chiffres dans l'ensemble d'entraînement.
\item Contenir de grandes variations ainsi que du bruit. Afin d'éviter le sur-apprentissage, il faut de nombreux exemples différents, voir très différents, les uns des autres ainsi que du bruit\footnote{Comme des faux exemples.}.
\end{itemize}

\subsection{\textit{Machine Learning} dans le cadre de la finance}
\subsection{Conclusion}
\newpage
%%%%%%%%%%%%%%%%%%%%%%%%%%%%%%%%%%%%%%%%%%%%%%%%%%%%%%%%%%%%%%%%%%%%%%%%%%
% début du projet  %%%%%%%%%%%%%%%%%%%%%%%%%%%%%%%%%%%%%%%%%%%%%%%%%%%%%%%
%%%%%%%%%%%%%%%%%%%%%%%%%%%%%%%%%%%%%%%%%%%%%%%%%%%%%%%%%%%%%%%%%%%%%%%%%%
\section{Projet}
\newpage
%%%%%%%%%%%%%%%%%%%%%%%%%%%%%%%%%%%%%%%%%%%%%%%%%%%%%%%%%%%%%%%%%%%%%%%%%%
% début de la bibliographie %%%%%%%%%%%%%%%%%%%%%%%%%%%%%%%%%%%%%%%%%%%%%%
%%%%%%%%%%%%%%%%%%%%%%%%%%%%%%%%%%%%%%%%%%%%%%%%%%%%%%%%%%%%%%%%%%%%%%%%%%
\section{Bibliographie}

\begin{enumerate}
\item Financial Times : "\textit{Real investors eclipsed by fast trading}", 2012 \url{https://www.ft.com/content/da5d033c-8e1c-11e1-bf8f-00144feab49a?mhq5j=e1} \label{real investors}
\item "\textit{A Machine Learning Approach to Automated Trading}", 09.05.2016, Ning Lu
\item "\textit{An efficient implementation of the backtesting of trading strategies.}" Ni, Jiarui, et Chegqi Zhang, \textit{Parallel and Distributed Processing and Applications} (2005): 126-131.
\item "\textit{Algorithmic Trading: Winning Strategies and Their Rationale ( Wiley Trading Series)}", John Wiley and Sons, 2013
\item "\textit{Machine Learning}", Mitchell, Tom M. New York, 1997. \label{mitchell}
\item Article Wikipédia sur SVM : \url{https://fr.wikipedia.org/wiki/Machine_\%C3\%A0_vecteurs_de_support} \label{wikipedia svm}
\item "\textit{Online Machine Learning Algorithms For Currency Exchange Prediction}", Eleftherios Soulas et Dennis Shasha de NYU, Courant Department. \label{descente du gradient stochastique}
\item  Article Wikipédia sur Algorithme du gradient : \url{https://fr.wikipedia.org/wiki/Algorithme_du_gradient} \label{wikipedia descente du gradient}
\item "\textit{Descision Tree Learning}", Tom M. Mitchell
\item Article Investopedia sur les Options \url{http://www.investopedia.com/terms/o/option.asp}
\item "\textit{Support Vector Machine (and Statistical Learning Theory) Tutorial}", de Jason Weston, NEC Labs America. \url{http://www.cs.columbia.edu/~kathy/cs4701/documents/jason_svm_tutorial.pdf}
\item  "\textit{An Introduction to Neural Networks}" Vincent Cheung et Kevin Cannons : \url{http://www2.econ.iastate.edu/tesfatsi/NeuralNetworks.CheungCannonNotes.pdf}
\item Exemple de réseaux de neurones\url{http://csc.lsu.edu/~jianhua/nn.pdf} p.5
\item \textit{Backtesting} Investopedia \url{http://www.investopedia.com/terms/b/backtesting.asp} \label{backtesting investopedia}
\item Historique des taux de changes : \url{http://www.xe.com/currencycharts/} \label{historique taux de change}
\item Article Wikipedia sur les Accords de Bretten Woods : \url{https://fr.wikipedia.org/wiki/Accords_de_Bretton_Woods} \label{wikipedia bretten woods}
\item Article Étalon-Or à Étalon-Dollar : \url{http://la-chronique-agora.com/etalon-or-etalon-dollar/} \label{étalon-or à étalon-dollar}
\item \textit{CHAPITRE 1 : LE MARCHE DES CHANGES Monnaie et Finance Internationales} de David Guerreiro, Université Paris 8, \url{https://economix.fr/docs/1045/chap_1_2015-16.pdf} \label{marche des changes}
\item Site \textit{FOREX} français : \url{http://www.forex.fr/newslist/8696-la-regle-des-trois-unites-du-marche-des-changes} \label{site fr forex}
\end{enumerate}


\end{document}