\documentclass[a4paper, 11pt]{article}

\usepackage[utf8]{inputenc}
\usepackage[T1]{fontenc}
\usepackage[francais]{babel} 
\usepackage{amsmath} % pour les formules de maths
\usepackage{amssymb} % pour des symboles
\usepackage{mathrsfs} % pour avoir acces a des jolies lettres calligrafiées. :)
\usepackage{listings} % pour le code source
\usepackage{color} % pour les couleurs
\usepackage{graphicx} % pour les graphiques (images)
\usepackage{fancyhdr} % pour utiliser le pagestyle fancy
\usepackage[headheight=14pt]{geometry} % pour les marges
\usepackage{hyperref}

\geometry{hmargin=3cm}

\title{Projet de Master}
\author{Romain Mencattini}
\date{\today}

\pagestyle{fancy} % pour avoir des entetes et des pieds de page
\renewcommand\headrulewidth{0.6pt}
\fancyhead[L]{Romain Mencattini} % haut de page gauche
\fancyhead[R]{Université de Genève \today} % haut de page droite

\begin{document}
\maketitle
\newpage
\tableofcontents
\newpage

%%%%%%%%%%%%%%%%%%%%%%%%%%%%%%%%%%%%%%%%%%%%%%%%%%%%%%%%%%%%%%%%%%%%%%%%%%
% début de l'état de l'art %%%%%%%%%%%%%%%%%%%%%%%%%%%%%%%%%%%%%%%%%%%%%%%
%%%%%%%%%%%%%%%%%%%%%%%%%%%%%%%%%%%%%%%%%%%%%%%%%%%%%%%%%%%%%%%%%%%%%%%%%%
\section{État de l'art}
\subsection{Introduction}
\subsection{Finance}
\subsection{Cadre théorique des algorithmes de \textit{Machine Learning}}
\subsection{\textit{Machine Learning} dans le cadre de la finance}
\subsection{Conclusion}
\newpage
%%%%%%%%%%%%%%%%%%%%%%%%%%%%%%%%%%%%%%%%%%%%%%%%%%%%%%%%%%%%%%%%%%%%%%%%%%
% début du projet  %%%%%%%%%%%%%%%%%%%%%%%%%%%%%%%%%%%%%%%%%%%%%%%%%%%%%%%
%%%%%%%%%%%%%%%%%%%%%%%%%%%%%%%%%%%%%%%%%%%%%%%%%%%%%%%%%%%%%%%%%%%%%%%%%%
\section{Projet}
\newpage
%%%%%%%%%%%%%%%%%%%%%%%%%%%%%%%%%%%%%%%%%%%%%%%%%%%%%%%%%%%%%%%%%%%%%%%%%%
% début de la bibliographie %%%%%%%%%%%%%%%%%%%%%%%%%%%%%%%%%%%%%%%%%%%%%%
%%%%%%%%%%%%%%%%%%%%%%%%%%%%%%%%%%%%%%%%%%%%%%%%%%%%%%%%%%%%%%%%%%%%%%%%%%
\section{Bibliographie}

\begin{enumerate}
\item Financial Times : "\textit{Real investors eclipsed by fast trading}", 2012 \url{https://www.ft.com/content/da5d033c-8e1c-11e1-bf8f-00144feab49a?mhq5j=e1}
\item "\textit{A Machine Learning Approach to Automated Trading}", 09.05.2016, Ning Lu
\item "\textit{An efficient implementation of the backtesting of trading strategies.}" Ni, Jiarui, et Chegqi Zhang, \textit{Parallel and Distributed Processing and Applications} (2005): 126-131.
\item "\textit{Algorithmic Trading: Winning Strategies and Their Rationale ( Wiley Trading Series)}", John Wiley and Sons, 2013
\item "\textit{Machine Learning}", Mitchell, Tom M. New York, 1997.
\item Article Wikipédia sur SVM : \url{https://fr.wikipedia.org/wiki/Machine_\%C3\%A0_vecteurs_de_support}
\item "\textit{Online Machine Learning Algorithms For Currency Exchange Prediction}", Eleftherios Soulas et Dennis Shasha de NYU, Courant Department.
\item  Article Wikipédia sur Algorithme du gradient : \url{https://fr.wikipedia.org/wiki/Algorithme_du_gradient}
\item "\textit{Descision Tree Learning}", Tom M. Mitchell
\item Article Investopedia sur les Options \url{http://www.investopedia.com/terms/o/option.asp}
\item "\textit{Support Vector Machine (and Statistical Learning Theory) Tutorial}", de Jason Weston, NEC Labs America. \url{http://www.cs.columbia.edu/~kathy/cs4701/documents/jason_svm_tutorial.pdf}
\item  "\textit{An Introduction to Neural Networks}" Vincent Cheung et Kevin Cannons : \url{http://www2.econ.iastate.edu/tesfatsi/NeuralNetworks.CheungCannonNotes.pdf}
\item Exemple de réseaux de neurones\url{http://csc.lsu.edu/~jianhua/nn.pdf} p.5
\end{enumerate}


\end{document}