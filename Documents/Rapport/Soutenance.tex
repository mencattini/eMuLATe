 \documentclass{beamer}

  \usepackage[utf8]{inputenc}
  \usepackage[T1]{fontenc}
  \usepackage{default}
  \usepackage{lmodern}
  \usepackage{color}
  \usepackage{listings}
  \usepackage{xcolor}
  \usepackage{caption}
  
  \DeclareCaptionFont{white}{\color{white}}
\DeclareCaptionFormat{listing}{\colorbox{gray}{\parbox{\textwidth}{#1#2#3}}}
\captionsetup[lstlisting]{format=listing,labelfont=white,textfont=white}

\definecolor{vert}{rgb}{0.2,0.65,0}

  \usetheme{Copenhagen}
    \definecolor{vert}{rgb}{0.2,0.65,0}
    
    \setbeamertemplate{headline}
{%
  \leavevmode%
  \begin{beamercolorbox}[wd=.5\paperwidth,ht=2.5ex,dp=1.125ex]{section in head/foot}%
    \hbox to .5\paperwidth{\hfil\insertsectionhead\hfil}
  \end{beamercolorbox}%
  \begin{beamercolorbox}[wd=.5\paperwidth,ht=2.5ex,dp=1.125ex]{subsection in head/foot}%
    \hbox to .5\paperwidth{\hfil\insertsubsectionhead\hfil}
  \end{beamercolorbox}%
}

\setbeamertemplate{navigation symbols}{} 

  \title{Soutenance Projet de Master}
  \author{Romain Mencattini}
  \institute{Université de Genève}
  \date{\today}
  
    \AtBeginSection[]
  {
    \begin{frame}
    \frametitle{Sommaire}
    \tableofcontents[currentsection, hideothersubsections, pausesubsections]
    \end{frame} 
  }

\begin{document}

  %%%%%%%%%%%%%%%%%%%%%%
  %%% Page de Garde %%%%
  %%%%%%%%%%%%%%%%%%%%%%
  \begin{frame}
  \titlepage
  \end{frame}

	\section{Introduction} % => ~ 2 + 6
	
	\begin{frame}
		\frametitle{Notions de finance}
		\begin{itemize}
			\item Définition FOREX (p.4)
			\item Principaux acteurs financiers (p.6)
			\item La règle des trois unités (p.6)
			\item Les outils possibles (spot, futur, option, p.6-7)
		\end{itemize}
	\end{frame}

	\begin{frame}
		\frametitle{Notions de \textit{Machine learning}}
		\begin{itemize}
			\item Définition du ML (p.7)
			\item Notions ensembles d'entraînement, de test,...
			\item Apprentissage supervisé et non supervisé
		\end{itemize}
	\end{frame}

	\begin{frame}
		\frametitle{Motivations}
		Aucun cross forex-machine learning (même si existe en action)
	\end{frame}
	
%%%%%%%%%%%%%%%%%%%%%%%%%%%%%%%%%%%%%%%%%%%%%%%%%%%%%%%%%%%%%%%%%%%%%%%%%%%%%
	\section{Algorithme et Implémentation}
	
	\begin{frame}
		\frametitle{Théorie \textit{Layer 1}}
		\begin{itemize}
			\item Équations layer 1
			\item Cela calcul un signal
			\item Prend en compte les $n$ derniers returns, la précédente réponse et d'autres paramètres
			\item Optimisation par descente du gradient (montrer équations)
		\end{itemize}
	\end{frame}

	\begin{frame}
		\frametitle{Théorie \textit{Layer 2}}
		\begin{itemize}
			\item Couche stop loss
			\item Expliquer le principe
			\item Montrer l'effet sur la courbe
		\end{itemize}
	\end{frame}

	\begin{frame}
		\frametitle{Théorie \textit{Layer 3}}
		\begin{itemize}
			\item Optimisation des meta-paramètres (faire parallèle avec le cours de choppard)
			\item Montrer équations
			\item Expliquer le principe et les choix de l'article
		\end{itemize}
	\end{frame}

	\begin{frame}
		\frametitle{Implémentation}
		Principalement des choix sur ce qui n'était pas expliqué
	\end{frame}

	\begin{frame}
		\frametitle{Problèmes}
		\begin{enumerate}
			\item Weights -> NaN
			\item Problème d'efficacité d'apprentissage
			\item Optimisation des meta-paramètres
			\item Vitesse général
		\end{enumerate}
	\end{frame}

	\begin{frame}
		\frametitle{Solutions}
		\begin{enumerate}
			\item RMSProp, AdaGrad, Adadelta
			\item utilisation du sharpe ratio
			\item recherche aléatoire normale centrée + diminution de certain paramètres recherchés
			\item Parallèlisme, éviter les listes etc.
		\end{enumerate}
	\end{frame}
	
%%%%%%%%%%%%%%%%%%%%%%%%%%%%%%%%%%%%%%%%%%%%%%%%%%%%%%%%%%%%%%%%%%%%%%%%%%%%%
	\section{Résultats}
	
	\begin{frame}
		\frametitle{Exemples de résultats}
		Exemple de 2000-2001, 2004-2006, 2006-2010 (EUR/USD)s
	\end{frame}

	\begin{frame}
		\frametitle{Vitesse d'exécution}
		Petit benchmark
	\end{frame}

%%%%%%%%%%%%%%%%%%%%%%%%%%%%%%%%%%%%%%%%%%%%%%%%%%%%%%%%%%%%%%%%%%%%%%%%%%%%%
	\section{Conclusion}
	
	\begin{frame}
		\frametitle{Pistes}
		Reprendre les pistes de mon rapport ou bien partir sur des méthodes structurelles (cf François Chollet).
	\end{frame}

	\begin{frame}
		\frametitle{Conclusion}
		Fonctionne localement mais pas globalement, nous sommes arrivés à qqch qui fonctionne à paritr de "rien", cela donne des pistes.
	\end{frame}
	

\end{document}
